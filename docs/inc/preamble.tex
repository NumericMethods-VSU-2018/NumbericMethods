%%% Преамбула %%%

\usepackage{fontspec} % XeTeX
\usepackage{xunicode} % Unicode для XeTeX

%\usepackage[T2A,T1]{fontenc}
%\usepackage[utf8]{inputenc}
%\usepackage[russian,english]{babel}

\defaultfontfeatures{Ligatures=TeX}
\setmainfont{Times New Roman} % Нормоконтроллеры хотят именно его
\newfontfamily\cyrillicfont{Times New Roman}

\usepackage{polyglossia}
\setdefaultlanguage{russian}

% Отступы у страниц
\usepackage{geometry}
\geometry{left=3cm}
\geometry{right=1.5cm}
\geometry{top=2cm}
\geometry{bottom=2cm}

\renewcommand{\baselinestretch}{1.5} % Полуторный межстрочный интервал
\setlength{\parindent}{1.27cm} % Абзацный отступ
\setlength{\parskip}{0.3cm} % Расстояние между абзацами

\setcounter{page}{2} % Начало нумерации страниц

\usepackage{indentfirst} % Красная строка после заголовка

%\usepackage[center]{titlesec} % Центрирование заголовков
%\titleformat{\section}
%  {\centering\large\bfseries} % format
%  {}% label
%  {0pt} % sep
%  {\large}

%\titleformat{\subsection}
%  {\normalsize\bfseries} % format
%  {}% label
%  {0pt} % sep
%  {\normalsize}

% Центрирование заголовков
\usepackage{sectsty}
\sectionfont{\centering}

% Списки
\usepackage{enumitem}
\setlist[enumerate,itemize]{leftmargin=12.7mm} % Отступы в списках

\makeatletter
    \AddEnumerateCounter{\asbuk}{\@asbuk}{м)}
\makeatother
\setlist{nolistsep} % Нет отступов между пунктами списка
